\documentclass{article}

\usepackage{amsmath}
\usepackage{graphicx}
\usepackage[table]{xcolor} % For row coloring

\title{Notes on IBEX and custom core}

\author{M. K. Dahl}

\begin{document}

\maketitle

\section{Introduction}

\section{Quick start}

\section{Notes About Simple System}

The simple system integrates the Ibex core as the host and connects to three slave devices. Below are the details of the system:

\subsection{Host}
The Ibex core acts as the host in the system. The host is defined as:
\begin{verbatim}
typedef enum logic {
    CoreD
} bus_host_e;
\end{verbatim}

\subsection{Slave Devices}
The system is integrated with three slave devices, which can be accessed using the following addresses:
\begin{itemize}
    \item \texttt{00}: RAM
    \item \texttt{01}: Simulation Control (\texttt{SimCtrl})
    \item \texttt{10}: Timer
\end{itemize}

The slave devices are defined as:
\begin{verbatim}
typedef enum logic[1:0] {
    Ram,
    SimCtrl,
    Timer
} bus_device_e;
\end{verbatim}

\subsection{RAM}
The RAM has the following characteristics:
\begin{itemize}
    \item Depth: \( \frac{1024 \times 1024}{4} = 256\text{k words} \), assuming each word is 4 bytes (32 bits).
    \item Total size: \( 1024 \times 1024 = 1\text{MB} \), divided by 4 to get 256k words.
\end{itemize}

The RAM file is set during simulation using the following command:
\begin{verbatim}
./build/lowrisc_ibex_ibex_simple_system_0/sim-verilator/Vibex_simple_system [-t] --meminit=ram,<sw_elf_file>
\end{verbatim}
Here, \texttt{<sw\_elf\_file>} should be the path to an ELF file (or alternatively a VMEM file).

The RAM is instantiated as:
\begin{verbatim}
ram_2p #(
    .Depth(1024*1024/4)
)
\end{verbatim}

The device address mapping for the RAM is configured as follows:
\begin{verbatim}
// Device address mapping
assign cfg_device_addr_mask[Ram] = ~32'hFFFFF; // 1 MB
\end{verbatim}
This masks the lower 20 bits, ensuring that everything above 12 bits is set to 0.

\subsection{Other Peripherals}
\begin{itemize}
    \item \textbf{BUS:} A simple bus that connects the host to the slave devices.
    \item \textbf{Simulation Control (\texttt{SimCtrl}):} A peripheral used to write ASCII output to a file.
    \item \textbf{Timer:} A basic timer with interrupt capabilities.
\end{itemize}

\subsection{Software Framework}
The system includes a software framework to interact with the peripherals and manage the system.

\section{Configurations}


The following items can be configured in the Ibex core, enabled or disabled. 
There are 4 support IBEX configs: small, opentitan, maxperf and maxperf-pmp-bmbalanced.

\begin{table}[h!]
    \centering
    \rowcolors{2}{gray!15}{white}
    \resizebox{\textwidth}{!}{
    \begin{tabular}{|l|p{7cm}|p{5cm}|l|}
    \hline
    \rowcolor{gray!30} \textbf{Name} & \textbf{Description} & \textbf{Possible Values} & \textbf{Value in Cille} \\ \hline
    RV32E & Determines whether the core uses the RV32E instruction set (16 general-purpose registers instead of 32). & 0 (disabled), 1 (enabled) & 0 \\ \hline
    RV32M & Configures the multiplier/divider extension. & \texttt{ibex\_pkg::RV32MNone}, \texttt{RV32MSingleCycle}, \texttt{RV32MFast} & \texttt{RV32MSingleCycle} \\ \hline
    RV32B & Configures the bit manipulation extension. & \texttt{ibex\_pkg::RV32BNone}, \texttt{RV32BOTEarlGrey} & \texttt{RV32BOTEarlGrey} \\ \hline
    RegFile & Type of register file used in the core. & \texttt{ibex\_pkg::RegFileFF}, \texttt{RegFileLatch} & \texttt{RegFileFF} \\ \hline
    BranchTargetALU & Enables a dedicated ALU for branch target calculation. & 0 (disabled), 1 (enabled) & 1 \\ \hline
    WritebackStage & Enables a separate writeback stage in the pipeline. & 0 (disabled), 1 (enabled) & 1 \\ \hline
    ICache & Enables the instruction cache. & 0 (disabled), 1 (enabled) & 1 \\ \hline
    ICacheECC & Enables error-correcting code (ECC) for the instruction cache. & 0 (disabled), 1 (enabled) & 1 \\ \hline
    ICacheScramble & Enables scrambling for instruction cache contents. & 0 (disabled), 1 (enabled) & 1 \\ \hline
    BranchPredictor & Enables the branch predictor for speculative execution. & 0 (disabled), 1 (enabled) & 1 \\ \hline
    DbgTriggerEn & Enables hardware debug triggers. & 0 (disabled), 1 (enabled) & 1 \\ \hline
    SecureIbex & Enables security features in the Ibex core. & 0 (disabled), 1 (enabled) & 1 \\ \hline
    PMPEnable & Enables the Physical Memory Protection (PMP) feature. & 0 (disabled), 1 (enabled) & 1 \\ \hline
    PMPGranularity & Defines the granularity of PMP regions (0 = default, higher values for finer granularity). & Integer values & 0 \\ \hline
    PMPNumRegions & Specifies the number of PMP regions available. & 0, 8, 16 & 16 \\ \hline
    MHPMCounterNum & Number of hardware performance monitoring counters. & 0, 4, 10 & 10 \\ \hline
    MHPMCounterWidth & Bit width of hardware performance monitoring counters. & 32, 64 & 32 \\ \hline
    \end{tabular}
    }
    \caption{Configuration options for the Cille configuration of the Ibex core.}
    \label{tab:cille_config}
\end{table}


\end{document}